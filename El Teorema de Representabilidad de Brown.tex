\documentclass{beamer}
\usetheme{Montpellier}
% Choose how your presentation looks.
%
% For more themes, color themes and font themes, see:
% http://deic.uab.es/~iblanes/beamer_gallery/index_by_theme.html
%
\mode<presentation>
{
  \usetheme{default}      % or try Darmstadt, Madrid, Warsaw, ...
  \usecolortheme{default} % or try albatross, beaver, crane, ...
  \usefonttheme{professionalfonts}  % or try serif, structurebold, ...
  \setbeamertemplate{navigation symbols}{}
  \setbeamertemplate{caption}[numbered]
} 

\usepackage[spanish]{babel}
\usepackage[utf8x]{inputenc}
\usepackage{graphicx}
\usepackage{amssymb}
\usepackage{amsfonts}
\usepackage{amsmath}
\usepackage{tikz-cd}

\newtheorem{teo}{Teorema}
\newtheorem{pro}{Proposici\'on}
\newtheorem{cor}{Corolario}
\newtheorem{lem}{Lema}
\newtheorem{df}{Definici\'on}
\newtheorem{ejem}{Ejemplo}
\newtheorem{obs}{Observaci\'on}
\newtheorem{no}{Notaci\'on}

\newcommand{\C}{\mathbf{C}}
\newcommand{\HT}{\mathbf{hTop}}
\newcommand{\T}{\mathbf{Top}}
\newcommand{\con}{\mathbf{Set}}
\newcommand{\G}{\mathbf{Ab}}
\newcommand{\Z}{\mathbb{Z}}

\DeclareGraphicsRule{.tif}{png}{.png}{`convert#1`diname#1`/basename#1.tif`.´png}

\title{El Teorema de Representabilidad de Brown}
\author{Luis Alberto Macías Barrales}
\institute{Instituto de Matemáticas, UNAM}
\date{Semestre 2020-2}

\begin{document}

\begin{frame}
\titlepage
\end{frame}

\begin{frame}{Indice}
\tableofcontents
\end{frame}


\section{Funtores representables}

\begin{frame}{Funtores representables}
	\begin{df}
	Sea $F\colon \mathbf{C}\to\mathbf{Set}$ un funtor contravariante, decimos que $F$ es representable si existe un objeto $C$ en $\mathbf{C}$ y una equivalencia natural $\varphi\colon F\to Mor_\mathbf{C} (\_ ,C)$. Al objeto $C$ lo llamamos objeto clasificante de $F$.
	\end{df}
	
	Notemos que por el lema de Yoneda, el objeto clasificante es único salvo isomorfismo.
\end{frame}

\begin{frame}{Nota histórica}
	(Agregar una nota histórica sobre los funtores representables)
\end{frame}

\begin{frame}{¿Qué información nos proporcionan los funtores?}
	
	\begin{ejem}
		 Sea $H^n \colon \mathbf{hW}\to \mathbf{Ab}$ el $n-$ésimo funtor de cohomología singular (con coeficientes en $\Z$). $H^n$ nos clasifica en cierto modo los \emph{agujeros} $n-$dimensionales de un espacio $X$. Tenemos la siguiente sucesión exacta:
			\[0 \to Ext_\Z (H_{n-1}(X),\Z)\to H^n(X) \to Hom_\Z (H_n (X),\Z)\to0\]
			Y si $X$ es $n-1$ conexo, utilizando el teorema general de Hurewicz, tenemos que $H_{n-1}(X)\cong 0$ y que $H_n (x)\cong \pi_n (X)$. Por lo que obtenemos un isomorfismo 
	\[H^n(X)\cong Hom_\Z(\pi_n(X),\Z)\]
	\end{ejem}
\end{frame}

\begin{frame}
	
		
\end{frame}




\end{document}
